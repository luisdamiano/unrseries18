% Yay!
% #000000 # 1 Negro
% #002269 # 2 Azul FIRST
% #CF102D # 3 Rojo FIRST
% #63656A # 4 Gris FIRST
% #E41F26 # 5 Rojo  secundario
% #2EA147 # 6 Verde secundario
% #1D79B4 # 7 Azul  secundario
% #E6E6E6 # 8 Gris  secundario

\definecolor{Negro}{HTML}{000000}
\definecolor{Blanco}{HTML}{FFFFFF}
\definecolor{Azul_FIRST}{HTML}{002269}
\definecolor{Rojo_FIRST}{HTML}{CF102D}
\definecolor{Gris_FIRST}{HTML}{63656A}
\definecolor{Rojo_secundario}{HTML}{E41F26}
\definecolor{Verde_secundario}{HTML}{2EA147}
\definecolor{Azul_secundario}{HTML}{1D79B4}
\definecolor{Gris_secundario}{HTML}{E6E6E6}

\setbeamercolor{frametitle}{bg = Verde_secundario}
\setbeamercolor{progress bar}{fg = Verde_secundario}
\setbeamercolor{title separator}{fg = Verde_secundario}

\setbeamerfont{footline}{size=\fontsize{9}{11}\selectfont}
\setbeamertemplate{footline}{
  \fontsize{4}{6}\selectfont

  \hfill
  Maestría en Estadística Aplicada, UNR
  \hfill
  |
  \hfill
  Series de Tiempo (2018)
  \hfill
  |
  \hfill
  \insertframenumber/\inserttotalframenumber
  \hfill
  \break
}

\setbeamerfont{footnote}{size=\tiny}

\usepackage{layouts}
\usepackage{tabularx}
\usepackage{booktabs}
\usepackage{bm}

\def\begincols{\begin{columns}}
\def\begincol{\begin{column}}
\def\endcol{\end{column}}
\def\endcols{\end{columns}}

\hypersetup{
  colorlinks,
  allcolors = .,
  urlcolor  = Verde_secundario,
}

% Reference size
\setbeamerfont{bibliography item}{size=\footnotesize}
\setbeamerfont{bibliography entry author}{size=\footnotesize}
\setbeamerfont{bibliography entry title}{size=\footnotesize}
\setbeamerfont{bibliography entry location}{size=\footnotesize}
\setbeamerfont{bibliography entry note}{size=\footnotesize}

% Reference break fix
% https://stackoverflow.com/a/35681531/2860744
% \widowpenalties{1}{150}

% Math commands Operators
\newcommand{\rank}[1]{\text{ran} \left( #1 \right)}
\newcommand{\med}[1]{\text{med} \left( #1 \right)}
\newcommand{\CI}[2]{\text{IC}_{#1} \left( #2 \right)}
\newcommand{\bias}[1]{\text{sesgo} \left( #1 \right)}
\newcommand{\cor}[2]{\text{corr} \left( #1, #2 \right)}
\newcommand{\var}[1]{\text{VaR} \left( #1 \right)}
\newcommand{\varr}[1]{\text{VR}}
\DeclareMathOperator{\evsym}{E}
\newcommand\ev[1]{\evsym\left\langle#1\right\rangle}
\DeclareMathOperator{\vsym}{V}
\newcommand\vv[1]{\vsym\left\langle#1\right\rangle}
\DeclareMathOperator{\covsym}{Cov}
\newcommand\cv[1]{\covsym\left\langle#1\right\rangle}
\DeclareMathOperator{\corrsym}{Corr}
\newcommand\corrv[1]{\corrsym\left\langle#1\right\rangle}
\newcommand\evs[2]{\evsym_{#1}\left\langle#2\right\rangle} % Expectation with subscript
\newcommand\vvs[2]{\vsym_{#1}\left\langle#2\right\rangle} % Variance with subscript
\DeclareMathOperator{\kusym}{Ku}
\newcommand\ku[1]{\kusym\left\langle#1\right\rangle}
%\newcommand\ind[1]{\mathcal{I}_{\left( #1 \right)}}
\newcommand\ind[1]{\mathbb{I}_{\left( #1 \right)}}
\newcommand\Real{\mathbb{R}}

\DeclareMathOperator*{\argmin}{arg\,min}
\DeclareMathOperator*{\argmax}{arg\,max}

\newcommand{\NN}{\mathcal{N}}

% https://tex.stackexchange.com/a/130584
\makeatletter
\newcommand\ChangeItemFont[3]{%
\renewcommand{\itemize}[1][]{%
  \beamer@ifempty{##1}{}{\def\beamer@defaultospec{#1}}%
  \ifnum \@itemdepth >2\relax\@toodeep\else
    \advance\@itemdepth\@ne
    \beamer@computepref\@itemdepth% sets \beameritemnestingprefix
    \usebeamerfont{itemize/enumerate \beameritemnestingprefix body}%
    \usebeamercolor[fg]{itemize/enumerate \beameritemnestingprefix body}%
    \usebeamertemplate{itemize/enumerate \beameritemnestingprefix body begin}%
    \list
      {\usebeamertemplate{itemize \beameritemnestingprefix item}}
      {\def\makelabel####1{%
          {%
            \hss\llap{{%
                \usebeamerfont*{itemize \beameritemnestingprefix item}%
                \usebeamercolor[fg]{itemize \beameritemnestingprefix item}####1}}%
          }%
        }%
  \ifnum\@itemdepth=1\relax
    #1%
  \else
  \ifnum\@itemdepth=2\relax
    #2%
  \else
  \ifnum\@itemdepth=3\relax
    #3%
  \fi%
  \fi%
  \fi%
  }
  \fi%
  \beamer@cramped%
  \raggedright%
  \beamer@firstlineitemizeunskip%
}}
\makeatother
